
\documentclass[twoside,twocolumn]{article}

\usepackage{blindtext} % Package to generate dummy text throughout this template 

\usepackage[sc]{mathpazo} % Use the Palatino font
\usepackage[T1]{fontenc} % Use 8-bit encoding that has 256 glyphs
\linespread{1.05} % Line spacing - Palatino needs more space between lines
\usepackage{microtype} % Slightly tweak font spacing for aesthetics

\usepackage[english]{babel} % Language hyphenation and typographical rules

\usepackage[hmarginratio=1:1,top=32mm,columnsep=20pt]{geometry} % Document margins
\usepackage[hang, small,labelfont=bf,up,textfont=it,up]{caption} % Custom captions under/above floats in tables or figures
\usepackage{booktabs} % Horizontal rules in tables

\usepackage{lettrine} % The lettrine is the first enlarged letter at the beginning of the text

\usepackage{enumitem} % Customized lists
\setlist[itemize]{noitemsep} % Make itemize lists more compact

\usepackage{abstract} % Allows abstract customization
\renewcommand{\abstractnamefont}{\normalfont\bfseries} % Set the "Abstract" text to bold
\renewcommand{\abstracttextfont}{\normalfont\small\itshape} % Set the abstract itself to small italic text

\usepackage{titlesec} % Allows customization of titles
\renewcommand\thesection{\Roman{section}} % Roman numerals for the sections
\renewcommand\thesubsection{\roman{subsection}} % roman numerals for subsections
\titleformat{\section}[block]{\large\scshape\centering}{\thesection.}{1em}{} % Change the look of the section titles
\titleformat{\subsection}[block]{\large}{\thesubsection.}{1em}{} % Change the look of the section titles

\usepackage{fancyhdr} % Headers and footers
\pagestyle{fancy} % All pages have headers and footers
\fancyhead{} % Blank out the default header
\fancyfoot{} % Blank out the default footer
\fancyhead[C]{Running title $\bullet$ May 2016 $\bullet$ Vol. XXI, No. 1} % Custom header text
\fancyfoot[RO,LE]{\thepage} % Custom footer text

\usepackage{titling} % Customizing the title section

\usepackage{hyperref} % For hyperlinks in the PDF

%----------------------------------------------------------------------------------------
%	TITLE SECTION
%----------------------------------------------------------------------------------------

\setlength{\droptitle}{-4\baselineskip} % Move the title up

\pretitle{\begin{center}\Huge\bfseries} % Article title formatting
\posttitle{\end{center}} % Article title closing formatting
\title{NEWTON'S RINGS USING LED EXPERIMENT KIT} % Article title
\author{%
\textsc{Vikram, Kiran} \\ %\thanks{A thank you or further information} \\[1ex] % Your name
\normalsize Indian Institute of Space Science and Technology \\ % Your institution
\normalsize \href{mailto:happykvng@gmail.com}{happykvng@gmail.com} % Your email address
%\and % Uncomment if 2 authors are required, duplicate these 4 lines if more
%\textsc{Jane Smith}\thanks{Corresponding author} \\[1ex] % Second author's name
%\normalsize University of Utah \\ % Second author's institution
%\normalsize \href{mailto:jane@smith.com}{jane@smith.com} % Second author's email address
}
\date{} % Leave empty to omit a date
\renewcommand{\maketitlehookd}{%
\begin{abstract}
1) To determine the radius of curvature of given plano-convex lens using yellow LED and Newton's rings.\\
2) To determine the unknown wavelength of the given LEDs by using Newton's rings and plano-convex lens of known radius of curvature and hence determining the energy bandgap of the given LED.
\end{abstract}
}

%----------------------------------------------------------------------------------------

\begin{document}

% Print the title
\maketitle

%----------------------------------------------------------------------------------------
%	ARTICLE CONTENTS
%----------------------------------------------------------------------------------------

\section{Aim}


1) To determine the radius of curvature of given plano-convex lens using yellow LED and Newton's rings.\\
2) To determine the unknown wavelength of the given LEDs by using Newton's rings and plano-convex lens of known radius of curvature and hence determining the energy bandgap of the given LED.



%------------------------------------------------
\section{Apparatus}

Travelling microscope, plano-convex lens, glass plates, yellow, green, blue and red LED modules with power supply, magnifying glass. 


%------------------------------------------------
\section{Theory}

The energy band gap E of a semiconductor is given by \\
\centerline{$ R  = \frac{D_{m}^{2}-D_{n}^{2}}{4 \lambda (m-n)}$} \\
\\ 
R = radius of curvature of plano convex lens\\
$D_{m}$ = Diameter of the $m^{th}$ ring\\
$D_{n}$ = Diameter of the $n^{th}$ ring\\
m,n = number of rings as measured from the center of the ring m > n\\
$\lambda$ = Wavelength of LED light source\\
\\
\centerline{$ E = \frac{hc}{\lambda}$}
Where E = bandgap of LED
h = plank's constant \\ 
c = speed of light\\

%------------------------------------------------

\section{Procedure}


\begin{itemize}
\item Setup the components.
\item Align the yello LED module with the stand such that the light if incident at the center of the inclined glass plate.
\item Make sure that the light coming from the LED should fall withing the area of the plane glass plate kept at 45 degree in wooden box. 
\item Count the rings from the center which is taken as the zeroth ring and go the left hand side until you reach the last visible ring. 
\item Note the readings on the horizontal scale of the Travelling microscope. 
\item Using the value of the wavelenth find the radius of curvature of the given planoconvex lens. 
\item Now without disturbing the TM replace the source with unknown wavelenght and repeate the procedure as in the last step. 
\item Find the wavelengths of red, green, blue LEDs and hence find the energy band gap for the LEDs. 
\end{itemize}

%------------------------------------------------

\section{Observations}

\begin{itemize}
\item Current C = 8.03 mA 
\item Distance between probes S = 0.24 cm
\item Thickness of sample W = 0.05 cm 
\end{itemize}
\begin{table}
\caption{Example table}
\centering
\begin{tabular}{llr}
\toprule
\multicolumn{2}{c}{Name} \\
\cmidrule(r){1-2}
First name & Last Name & Grade \\
\midrule
John & Doe & $7.5$ \\
Richard & Miles & $2$ \\
\bottomrule
\end{tabular}
\end{table}

\blindtext % Dummy text

\begin{equation}
\label{eq:emc}
e = mc^2
\end{equation}

\blindtext % Dummy text

%------------------------------------------------
\section{Procedure}


\begin{itemize}
\item Put the four probe arrangement in the oven and connect the lead of the oven to socket(10). Also insert a PT100 temperature sensor into the hole given at the top of the four probe arrangement. 
\item Conncet the red and black plug leads of the four probe arrangement to 4mm sockets marked as voltage V. 
\item Conncet the yello plug leads to 4mm sockets marked as current. 
\item Change swith to current mode to display the current reading. 
\item Switch on the apparatus and slightly increase the current using current knob, say 4mA and note that the voltage should be positive.
\item Set the current to desired value say 8 mA using current adjusting knob. Also select the range of multiplier using switch to x1 or x10.
\item Switch on the oven. Green LED will glow showing the oven is on. 
\item Change switch to temperature for display to show temperature reading. 
\item Note the probe voltage on display for different values of temperatures.
\end{itemize}

%------------------------------------------------
%------------------------------------------------

\section{Discussion}

\subsection{Subsection One}

A statement requiring citation \cite{Figueredo:2009dg}.
\blindtext % Dummy text

\subsection{Subsection Two}

\blindtext % Dummy text

%----------------------------------------------------------------------------------------
%	REFERENCE LIST
%----------------------------------------------------------------------------------------

\begin{thebibliography}{99} % Bibliography - this is intentionally simple in this template

\bibitem[Figueredo and Wolf, 2009]{Figueredo:2009dg}
Figueredo, A.~J. and Wolf, P. S.~A. (2009).
\newblock Assortative pairing and life history strategy - a cross-cultural
  study.
\newblock {\em Human Nature}, 20:317--330.
 
\end{thebibliography}

%----------------------------------------------------------------------------------------

\end{document}
