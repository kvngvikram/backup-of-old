%%%%%%%%%%%%%%%%%%%%%%%%%%%%%%%%%%%%%%%%%
% Journal Article
% LaTeX Template
% Version 1.4 (15/5/16)
%
% This template has been downloaded from:
% http://www.LaTeXTemplates.com
%
% Original author:
% Frits Wenneker (http://www.howtotex.com) with extensive modifications by
% Vel (vel@LaTeXTemplates.com)
%
% License:
% CC BY-NC-SA 3.0 (http://creativecommons.org/licenses/by-nc-sa/3.0/)
%
%%%%%%%%%%%%%%%%%%%%%%%%%%%%%%%%%%%%%%%%%

%----------------------------------------------------------------------------------------
%	PACKAGES AND OTHER DOCUMENT CONFIGURATIONS
%----------------------------------------------------------------------------------------

\documentclass[twoside,twocolumn]{article}

\usepackage{blindtext} % Package to generate dummy text throughout this template 

\usepackage[sc]{mathpazo} % Use the Palatino font
\usepackage[T1]{fontenc} % Use 8-bit encoding that has 256 glyphs
\linespread{1.05} % Line spacing - Palatino needs more space between lines
\usepackage{microtype} % Slightly tweak font spacing for aesthetics

\usepackage[english]{babel} % Language hyphenation and typographical rules

\usepackage[hmarginratio=1:1,top=32mm,columnsep=20pt]{geometry} % Document margins
\usepackage[hang, small,labelfont=bf,up,textfont=it,up]{caption} % Custom captions under/above floats in tables or figures
\usepackage{booktabs} % Horizontal rules in tables

\usepackage{lettrine} % The lettrine is the first enlarged letter at the beginning of the text

\usepackage{enumitem} % Customized lists
\setlist[itemize]{noitemsep} % Make itemize lists more compact

\usepackage{abstract} % Allows abstract customization
\renewcommand{\abstractnamefont}{\normalfont\bfseries} % Set the "Abstract" text to bold
\renewcommand{\abstracttextfont}{\normalfont\small\itshape} % Set the abstract itself to small italic text

\usepackage{titlesec} % Allows customization of titles
\renewcommand\thesection{\Roman{section}} % Roman numerals for the sections
\renewcommand\thesubsection{\roman{subsection}} % roman numerals for subsections
\titleformat{\section}[block]{\large\scshape\centering}{\thesection.}{1em}{} % Change the look of the section titles
\titleformat{\subsection}[block]{\large}{\thesubsection.}{1em}{} % Change the look of the section titles

\usepackage{fancyhdr} % Headers and footers
\pagestyle{fancy} % All pages have headers and footers
\fancyhead{} % Blank out the default header
\fancyfoot{} % Blank out the default footer
\fancyhead[C]{Running title $\bullet$ May 2016 $\bullet$ Vol. XXI, No. 1} % Custom header text
\fancyfoot[RO,LE]{\thepage} % Custom footer text

\usepackage{titling} % Customizing the title section

\usepackage{hyperref} % For hyperlinks in the PDF

%----------------------------------------------------------------------------------------
%	TITLE SECTION
%----------------------------------------------------------------------------------------

\setlength{\droptitle}{-4\baselineskip} % Move the title up

\pretitle{\begin{center}\Huge\bfseries} % Article title formatting
\posttitle{\end{center}} % Article title closing formatting
\title{FOUR PROBE APPARATUS} % Article title
\author{%
\textsc{Vikram, Kiran} \\ %\thanks{A thank you or further information} \\[1ex] % Your name
\normalsize Indian Institute of Space Science and Technology \\ % Your institution
\normalsize \href{mailto:happykvng@gmail.com}{happykvng@gmail.com} % Your email address
%\and % Uncomment if 2 authors are required, duplicate these 4 lines if more
%\textsc{Jane Smith}\thanks{Corresponding author} \\[1ex] % Second author's name
%\normalsize University of Utah \\ % Second author's institution
%\normalsize \href{mailto:jane@smith.com}{jane@smith.com} % Second author's email address
}
\date{} % Leave empty to omit a date
\renewcommand{\maketitlehookd}{%
\begin{abstract}
%\noindent \blindtext % Dummy abstract text - replace \blindtext with your %abstract text
hello ! how are you my pussy kat we love pussy cat. and also dogs are man's best friends 
\end{abstract}
}

%----------------------------------------------------------------------------------------

\begin{document}

% Print the title
\maketitle

%----------------------------------------------------------------------------------------
%	ARTICLE CONTENTS
%----------------------------------------------------------------------------------------

\section{Introduction}

\lettrine[nindent=0em,lines=3]{T}he purpose of 4 point probe method is to measure the resistivity of a semiconductor sample. The 4 probe setup consists of 4 equally spaced metal tips with finite radius. The tip is supported by springs on the other end to minimize excessive pressure on the metal. A high impedence current source is used to supply current through the outer probes. A voltmeter measures the voltage across the inner probes to determin the sample resistivity and hence, the energy band gap of the semiconductor sample.
The sharp probes are placed on a 





%------------------------------------------------
\section{Apparatus}

SK012 FOUR PROBE APPARATUS, Semiconductor sample. 





%------------------------------------------------
\section{Theory}

The energy band gap E of a semiconductor is given by \\
\centerline{$ E_{g}  = \frac{2k \times  2.3026 \times log_{10}\rho }{T^{-1}} \hspace{0.5cm} eV $} \\
\\ 
Where $ k = 8.6 \times 10^{-5} \hspace{0.5cm} eV/deg$ .\\
$\rho$ is the resistivity of the semiconductor sample given by $\rho = \rho_{0} / f(W/s)$\\
\\
Where $\rho_{0} = \frac{V \times 2\pi s}{I}$\\
W is the thickness of the sample.\\ s is the probe spacing.\\ f(W/s) function is the correction factor.\\ V is the voltage across the two inner probes.\\ I is the current across the two inner probes.\\


%------------------------------------------------

\section{Procedure}


\begin{itemize}
\item Put the four probe arrangement in the oven and connect the lead of the oven to socket(10). Also insert a PT100 temperature sensor into the hole given at the top of the four probe arrangement. 
\item Conncet the red and black plug leads of the four probe arrangement to 4mm sockets marked as voltage V. 
\item Conncet the yello plug leads to 4mm sockets marked as current. 
\item Change swith to current mode to display the current reading. 
\item Switch on the apparatus and slightly increase the current using current knob, say 4mA and note that the voltage should be positive.
\item Set the current to desired value say 8 mA using current adjusting knob. Also select the range of multiplier using switch to x1 or x10.
\item Switch on the oven. Green LED will glow showing the oven is on. 
\item Change switch to temperature for display to show temperature reading. 
\item Note the probe voltage on display for different values of temperatures.
\end{itemize}

%------------------------------------------------

\section{Observations}

\begin{itemize}
\item Current C = 8.03 mA 
\item Distance between probes S = 0.24 cm
\item Thickness of sample W = 0.05 cm 
\end{itemize}
\begin{table}
\caption{Example table}
\centering
\begin{tabular}{llr}
\toprule
\multicolumn{2}{c}{Name} \\
\cmidrule(r){1-2}
First name & Last Name & Grade \\
\midrule
John & Doe & $7.5$ \\
Richard & Miles & $2$ \\
\bottomrule
\end{tabular}
\end{table}

\blindtext % Dummy text

\begin{equation}
\label{eq:emc}
e = mc^2
\end{equation}

\blindtext % Dummy text

%------------------------------------------------
\section{Procedure}


\begin{itemize}
\item Put the four probe arrangement in the oven and connect the lead of the oven to socket(10). Also insert a PT100 temperature sensor into the hole given at the top of the four probe arrangement. 
\item Conncet the red and black plug leads of the four probe arrangement to 4mm sockets marked as voltage V. 
\item Conncet the yello plug leads to 4mm sockets marked as current. 
\item Change swith to current mode to display the current reading. 
\item Switch on the apparatus and slightly increase the current using current knob, say 4mA and note that the voltage should be positive.
\item Set the current to desired value say 8 mA using current adjusting knob. Also select the range of multiplier using switch to x1 or x10.
\item Switch on the oven. Green LED will glow showing the oven is on. 
\item Change switch to temperature for display to show temperature reading. 
\item Note the probe voltage on display for different values of temperatures.
\end{itemize}

%------------------------------------------------
%------------------------------------------------

\section{Discussion}

\subsection{Subsection One}

A statement requiring citation \cite{Figueredo:2009dg}.
\blindtext % Dummy text

\subsection{Subsection Two}

\blindtext % Dummy text

%----------------------------------------------------------------------------------------
%	REFERENCE LIST
%----------------------------------------------------------------------------------------

\begin{thebibliography}{99} % Bibliography - this is intentionally simple in this template

\bibitem[Figueredo and Wolf, 2009]{Figueredo:2009dg}
Figueredo, A.~J. and Wolf, P. S.~A. (2009).
\newblock Assortative pairing and life history strategy - a cross-cultural
  study.
\newblock {\em Human Nature}, 20:317--330.
 
\end{thebibliography}

%----------------------------------------------------------------------------------------

\end{document}
\grid
